\documentclass[]{iac}

\DeclareMathOperator{\E}{E}
\DeclareMathOperator{\prob}{p}
\DeclareMathOperator{\tr}{tr}

\newcommand{\etalia}{\textit{et al.}}
\newcommand*{\vectornorm}[1]{\left\|#1\right\|}
\newcommand*\rfrac[2]{{{}^{#1}\!/_{#2}}} % running fraction with slash - requires math mode.
\newcommand*\T{\mathsf{T}}

\begin{document}

\IACpaperyear{17}
\IACpapernumber{D9.2.8}
\IACconference{65}
\IAClocation{Toronto, Canada}
\IACcopyrightA{2014}{the authors}

\title{Technology demonstrator of a rocket carrying a deployable fleet of autonomous gliders}

%\author{Main~Author\\Affiliation, Country, email address\and
%		Co-Author\\Affiliation, Country, email address}
\IACauthor{Patrick Spieler}{Swiss Federal Institute of Technology, Lausanne, Switzerland, }
\IACauthor{Co-Author}{Affiliation, Country, email address}

\abstract{%Rewrite: Dalmir, Sorina
The Intercollegiate Rocket Competition (IREC) aims to gather students from across the world to design and build a rocket that reaches an altitude of 3km or 10km, carrying a 4kg of payload. As part of this competition, the Team Duster, formed by students from Swiss universities developed a rocket with a payload targeting an apogee of 3km. \\ 
Firstly, the rocket flying to 3km will be presented, along with its design and manufacturing processes. The rocket then follows a dual-event recovery process. Firstly, the drogue (small) parachute is deployed, reducing the speed of the rocket to about 30 m/s. At the same time the separation of the nosecone occurs thereby releasing the payload. After the rocket descends to an altitude of 457 meters, the second (main) parachute is deployed, reducing the speed to 6m/s. Throughout the launch, separation, and landing phases, we constantly received telemetry (information) provided by the custom-designed avionic components placed in the nosecone of the rocket. To ensure that the deployment occurs at predefined altitudes, the decision was made to use 2 redundant systems for altitude measurement -independent of the avionics placed in the recovery bay. The trajectory of the rocket was simulated in 3 varying environments, including our own developed simulator. The rocket made its first flight towards the end of March 2017 in Switzerland where its flight data was compared with the simulation data. Based on this data, the necessary trajectory corrections were performed in order to improve the competition Flight - which occured in mid-June. \\
Secondly, an innovative payload that flew in the rocket will also be presented. A fleet of 3 gliders are deployed from the rocket at apogee using a release based on a spring mechanism. Equipped with an autopilot, differential GPS, and control surfaces using servomotors, the gliders autonomously fly in formation. Eventually, they land at a fixed point on the ground. The gliders have a wing span of approximately 200mm and a body length of 100mm. During the flight, the gliders transmit information back to the groundstation and are tracked using custom-made ground station.  Potential video-recording of the flight will be investigated in the future.
The results of the intermediary flight (end-of-March) as well as the competition flight -both in terms of rocket trajectory and flight of the gliders- is included in the paper, along with further recommendations for a more advanced technology demonstration in the future.}

\maketitle
%Check for number of pages allowed, if any
\section{Introduction-DALMIR}

1. Explain the project

2. Explain what we did for the project and why?

3. Explain how the paper is organized.



\section{Rocket}

\subsection{General Overview}



\subsection{Design and Manufacturing}
%Under design, mention static & dynamic stability(PATRICK) 



\subsection{Recovery}

\subsection{Avionics}

%Podium session material
% Last 2 pictures (overall system explanation with arrows)

\subsection{Flight Tests-PATRICK}


\section{Gliders}

\subsection{General Overview}

\subsection{Design and Manufacturing of ONE glider-SORINA}
%Pictures from report here

\subsection{Design and Manufacturing of a fleet of gliders-SORINA}
%Concept TBD in solidworks 

\subsection{Navigation and Control of the fleet of gliders}
%Ultrawide beacons 
%Formation
%Check for alternative options, instead of GPS


\section{Conclusions}
\section{Outlook}
%Maybe talk about airbrakes, active control, etc.

\end{document}